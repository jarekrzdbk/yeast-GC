\documentclass{article}

\usepackage[pdftex]{graphicx}

\title{Analysis of gc count ratio in 3 strains of yeast}
\author{Yaroslav Rozdobudko}

\begin{document}

\maketitle

\section{Introduction}

\begin{verbatim}
    Analysis contains 3 strains of yeast
    CEN.PK113-7D_Delft_2012_AEHG00000000
    EC9-8_ASinica_2011_AGSJ01000000
    FL100_SGD_2015_JRIT00000000
\end{verbatim}




\section{Methods}
Yeast genome was subject of interest for some time ~\cite{DUJON1996263} ~\cite{Dujon2004} ~\cite{Dujon2017}
In this repository solution was developed to calculate GC ratio in ORFs of 3 yeast strains.
First fasta files for unfolded using fasta-unfold script, then content of unfolded files was passed to find-orfs script,
which looked for ORFs, then GC content ratio were calculated.
generate-plot.R script generated plots from gc files
resulting plots look similar to plots presented in research done by Lynch ~\cite{Lynch2010} 

\section{Plots}

\begin{figure}
    \includegraphics[scale=0.4]{outputs/cen_gc_frequency_ratio.pdf}
    \includegraphics[scale=0.4]{outputs/fl_gc_frequency_ratio.pdf}
    \includegraphics[scale=0.4]{outputs/ec_gc_frequency_ratio.pdf}
    \caption{GC frequency ratio histograms}
\end{figure}

\begin{figure}
    \includegraphics[scale=0.8]{outputs/combined_histogram.pdf}
    \caption{combined frequency ratio histograms}
\end{figure}

\begin{figure}
    \includegraphics[scale=0.8]{outputs/combined_boxplot.pdf}
    \caption{combined frequency ratio boxplot}
\end{figure}

\begin{figure}
    \includegraphics[scale=0.8]{outputs/combined_densityplot.pdf}
    \caption{combined frequency ratio densityplot}
\end{figure}

\clearpage
\bibliography{bibliography}
\bibliographystyle{plain}

\end{document}
